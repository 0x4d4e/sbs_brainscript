\documentclass{beamer}

\usepackage[utf8x]{inputenc}

\usetheme{default}

\title[Brainscript]{Brainscript}
\subtitle{A Postscript interpreter and visualizer for Brainfuck programs}
\author{Fabian Grünbichler, Matthias Neumayr}


\begin{document}

\frame{
    \titlepage
}


\begin{frame}{Brainfuck?}

    \begin{itemize}
        \item Esoterische Programmiersprache
        \item \emph{Turing Tarpit}
        \item Datenspeicher + Pointer auf aktuelle Speicherzelle
        \item Programmspeicher + Pointer auf aktuelle Instruktion
    \end{itemize}

\end{frame}

\begin{frame}{8 Befehle}

    \begin{center}
        Daten- und Instruktionspointer zu Beginn $0$, Datenspeicher leer\\ 
        \begin{table}
            \begin{tabular}[h!]{c|l}
                $>$ & Daten-Pointer eine Position nach rechts verschieben (d++) \\ 
                $<$ & Daten-Pointer eine Position nach links verschieben (d--) \\
                $+$ & Inhalt der aktuellen Speicherzelle um 1 erhöhen (*d++) \\
                $-$ & Inhalt der aktuellen Speicherzelle um 1 verringern (*d--) \\
                . & Inhalt der aktuellen Speicherzelle als ASCII-Zeichen ausgeben \\
                , & 1 Byte Input einlesen und in aktuelle Speicherzelle speichern \\
                $[$ & Schleifenbeginn, Bedingung aktuelles Datenelement  $= 0$ \\
                $]$ & Schleifenende, Sprung zu Beginn wenn aktuelles Datenelement $\neq 0$ \\
            \end{tabular}
        \end{table}
    \end{center}

\end{frame}

\begin{frame}[fragile,c]{Beispielcode}

    \emph{Matthias} ausgeben:

    \begin{verbatim}
++++++++++++++++++++++++++++++++++++++++++++++++++++
+++++++++++++++++++++++++.++++++++++++++++++++.+++++
++++++++++++++..------------.+.--------.++++++++++++
++++++.
    \end{verbatim}

\end{frame}

\begin{frame}[fragile,c]{Beispielcode ctd}

    Fibonacci-Zahlen bis $100$ berechnen:

    \begin{verbatim}
+++++++++++>+>>>>+++++++++++++++++++++++++++++++++++
+++++++++>++++++++++++++++++++++++++++++++<<<<<<[>[>
>>>>>+>+<<<<<<<-]>>>>>>>[<<<<<<<+>>>>>>>-]<[>+++++++
+++[-<-[>>+>+<<<-]>>>[<<<+>>>-]+<[>[-]<[-]]>[<<[>>>+
<<<-]>>[-]]<<]>>>[>>+>+<<<-]>>>[<<<+>>>-]+<[>[-]<[-]
]>[<<+>>[-]]<<<<<<<]>>>>>[++++++++++++++++++++++++++
++++++++++++++++++++++.[-]]++++++++++<[->-<]>+++++++
+++++++++++++++++++++++++++++++++++++++++.[-]<<<<<<<
<<<<<[>>>+>+<<<<-]>>>>[<<<<+>>>>-]<-[>>.>.<<<[-]]<<[
>>+>+<<<-]>>>[<<<+>>>-]<<[<+>-]>[<+>-]<<<-]
    \end{verbatim}

\end{frame}

\begin{frame}{Aktueller Stand}

    \begin{itemize}
        \item Interpreter
        \item Programm aus Datei oder als String vom Stack lesen
        \item Instruktionsweises Ausführen des Programms
        \item Input von stdin lesen (immer nur 1 Char)
        \item Output ausgeben (stdout und Array)
    \end{itemize}

\end{frame}

\begin{frame}{TODO - Visualisierung, Wrapper}

    \begin{itemize}
        \item PS-Output: Instruktionsspeicher + aktuelle Position
        \item PS-Output: Datenspeicher + aktuelle Position
        \item PS-Output: bisheriger Output
        \item Wrapper (bash,C?) für komfortable, interaktive Kontrolle
        \item (mehr) Fehlerbehandlung
    \end{itemize}

\end{frame}

\end{document}
