\documentclass{beamer}

\usepackage[utf8x]{inputenc}

\usetheme{default}

\title[Brainscript]{Brainscript}
\subtitle{A Postscript interpreter and visualizer for Brainfuck programs}
\author{Fabian Grünbichler, Matthias Neumayr}


\begin{document}

\frame{
    \titlepage
}


\begin{frame}{Brainfuck?}

    \begin{itemize}
        \item Esoterische Programmiersprache
        \item \emph{Turing Tarpit}
        \item Datenspeicher + Pointer auf aktuelle Speicherzelle
        \item Programmspeicher + Pointer auf aktuelle Instruktion
    \end{itemize}

\end{frame}

\begin{frame}{8 Befehle}

    \begin{center}
        Daten- und Instruktionspointer zu Beginn $0$, Datenspeicher leer\\ 
        \begin{table}
            \begin{tabular}[h!]{c|l}
                $>$ & Daten-Pointer eine Position nach rechts ($d++$) \\ 
                $<$ & Daten-Pointer eine Position nach links ($d--$) \\
                $+$ & Wert in aktueller Speicherzelle um 1 erhöhen ($*d++$) \\
                $-$ & Wert in aktueller Speicherzelle um 1 verringern ($*d--$) \\
                . & Inhalt der aktuellen Speicherzelle als ASCII-Zeichen ausgeben \\
                , & 1 Byte Input einlesen und in aktuelle Speicherzelle speichern \\
                $[$ & Schleifenbeginn, while  $d \neq 0$ \\
                $]$ & Schleifenende, unbedingter Sprung zu Beginn \\
            \end{tabular}
        \end{table}
    \end{center}

\end{frame}


\begin{frame}[fragile,c]{Beispielcode}

    \emph{Hallo World!} ausgeben:

    \begin{verbatim}
+++++ +++++             initialize counter (cell #0) to 10
[   > +++++ ++              add  7 to cell #1
    > +++++ +++++           add 10 to cell #2 
    > +++                   add  3 to cell #3
    > +                     add  1 to cell #4
    <<<< -  ]           decrement counter (cell #0)           
> ++ . > + .            'He'            
+++++ ++ . .            'll'
+++ . > ++ .            'o '
<< +++++ +++++ +++++ .  'W'
> . +++ .               'or'
----- - .               'l'
----- --- .             'd'
> + . > .               '!\n'
    \end{verbatim}

    400 Instructions

\end{frame}


\begin{frame}[fragile,c]{Beispielcode ctd}

    Fibonacci-Zahlen bis $100$ berechnen:

    \begin{verbatim}
+++++++++++>+>>>>+++++++++++++++++++++++++++++++++++
+++++++++>++++++++++++++++++++++++++++++++<<<<<<[>[>
>>>>>+>+<<<<<<<-]>>>>>>>[<<<<<<<+>>>>>>>-]<[>+++++++
+++[-<-[>>+>+<<<-]>>>[<<<+>>>-]+<[>[-]<[-]]>[<<[>>>+
<<<-]>>[-]]<<]>>>[>>+>+<<<-]>>>[<<<+>>>-]+<[>[-]<[-]
]>[<<+>>[-]]<<<<<<<]>>>>>[++++++++++++++++++++++++++
++++++++++++++++++++++.[-]]++++++++++<[->-<]>+++++++
+++++++++++++++++++++++++++++++++++++++++.[-]<<<<<<<
<<<<<[>>>+>+<<<<-]>>>>[<<<<+>>>>-]<-[>>.>.<<<[-]]<<[
>>+>+<<<-]>>>[<<<+>>>-]<<[<+>-]>[<+>-]<<<-]
    \end{verbatim}

    183510 Instructions

\end{frame}

\begin{frame}{Aktueller Stand}

    \begin{itemize}
        \item Interpreter
        \item Programm aus Datei oder als String vom Stack lesen
        \item Instruktionsweises Ausführen des Programms
        \item X Instruktionen ausführen
        \item Code bis zum Ende ausführen
        \item Input von stdin lesen (immer nur 1 Char)
    \end{itemize}

\end{frame}

\begin{frame}{Visualisierung}

    \begin{itemize}
        \item Datenspeicherausschnitt (``Tape'')
        \item Codeausschnitt
        \item Aktueller Status \\Instructioncounter, Tape- und Codeposition, Fehlermeldung
        \item Programm-Output
    \end{itemize}
\end{frame}

\begin{frame}{Sprachspezifische Features}
    \begin{itemize}
        \item Mapping zw. BF-Instruktionen und PS-Prozeduren mit Dictionary
        \item "generische" Methoden zum Zeichnen von Tape und Code
        \item kurze Prozeduren, 4 wichtigsten Werte oben am Stack
    \end{itemize}
\end{frame}
\end{document}
